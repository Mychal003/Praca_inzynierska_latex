\section*{Streszczenie}

Przedstawiona praca inżynierska zawiera projekt i wdrożenie systemu RAG (Retrieval-Augmented Generation) do generowania automatycznych odpowiedzi. Odpowiadając na pytania użytkowników, system na podstawie dokumentacji technicznej łączy wyszukiwanie informacji z generowaniem odpowiedzi przy użyciu dużych modeli językowych (LLM).

Opracowane rozwiązanie przetwarza pliki w formacie PDF, dzieli je na fragmenty (chunki), tworzy reprezentacje wektorowe (embeddingi) i zapisuje je w bazie FAISS. Po otrzymaniu zapytania użytkownika system wyszukuje najbardziej pasujące fragmenty dokumentu, \\a następnie generuje odpowiedź za pomocą modelu GPT-4o.

W ramach przedstawionej pracy inżynierskiej przeprowadzono ewaluację systemu, wykorzystując metryki wyszukiwania (Precision, Recall, MRR, NDCG) oraz metryki generacji (ROUGE, Semantic Similarity, LLM Judge).

Najważniejsza konkluzja przeprowadzonej ewaluacji pozwala przypuszczać, że odpowiednie projektowanie promptów (prompt engineering) odegrało kluczową rolę w eliminacji błędów \\i zapewnieniu poprawności merytorycznej odpowiedzi. Zoptymalizowany system osiągnął 93.8\% ogólnej jakości według oceny LLM Judge, przy zachowaniu 100\% skuteczności \\w udzielaniu odpowiedzi.