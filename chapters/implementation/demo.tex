W tej sekcji przedstawiono demonstrację działania systemu na 
przykładzie dokumentacji routera TP-Link Archer C7.

\subsection{Logowanie do systemu}

Po uruchomieniu aplikacji użytkownik widzi ekran logowania (Rysunek~\ref{fig:demo_login}). Nowi użytkownicy mogą założyć konto przechodząc do zakładki 
,,Rejestracja''. Po zalogowaniu system przekierowuje do głównego 
interfejsu aplikacji.

\begin{figure}[H]
\centering
\includegraphics[width=0.7\textwidth]{images/implementation/login_screen.png}
\caption{Ekran logowania do systemu}
\label{fig:demo_login}
\end{figure}

\subsection{Tworzenie konwersacji i przesyłanie dokumentu}

Po zalogowaniu użytkownik widzi panel boczny z listą swoich 
konwersacji. Aby rozpocząć nową rozmowę, należy kliknąć przycisk 
,,+'' w panelu bocznym, a następnie przesłać dokument PDF.

Proces przetwarzania dokumentu obejmuje ekstrakcję tekstu, 
podział na chunki oraz generowanie embeddingów. Czas 
przetwarzania dla typowej dokumentacji technicznej wynosi kilka sekund.

\newpage
\subsection{Zadawanie pytań}

Po przetworzeniu dokumentu użytkownik może zadawać pytania 
w języku naturalnym. System automatycznie klasyfikuje typ 
pytania i dostosowuje format odpowiedzi:

\begin{itemize}
    \item \textbf{Pytania faktyczne} -- zwięzła, konkretna informacja
    \item \textbf{Pytania proceduralne} -- numerowana lista kroków
    \item \textbf{Pytania o problemy} -- diagnoza i propozycje rozwiązań
\end{itemize}

Rysunek~\ref{fig:demo_factual} przedstawia przykład odpowiedzi 
na pytanie faktyczne dotyczące specyfikacji urządzenia.

\begin{figure}[H]
\centering
\includegraphics[width=0.9\textwidth]{images/implementation/demo_factual.png}
\caption{Przykład odpowiedzi na pytanie faktyczne}
\label{fig:demo_factual}
\end{figure}

\newpage
Rysunek~\ref{fig:demo_procedural} przedstawia odpowiedź na pytanie 
proceduralne z instrukcją krok po kroku.

\begin{figure}[H]
\centering
\includegraphics[width=0.9\textwidth]{images/implementation/demo_procedural.png}
\caption{Przykład odpowiedzi na pytanie proceduralne}
\label{fig:demo_procedural}
\end{figure}

\newpage
\subsection{Wyświetlanie źródeł}

Każda odpowiedź zawiera rozwijalną sekcję ,,Źródła'' -- fragmenty 
dokumentu użyte do wygenerowania odpowiedzi. 
Rysunek~\ref{fig:demo_sources} przedstawia przykład wyświetlonych źródeł.

\begin{figure}[H]
\centering
\includegraphics[width=0.9\textwidth]{images/implementation/demo_sources1.png}
\caption{Źródła użyte do wygenerowania odpowiedzi}
\label{fig:demo_sources}
\end{figure}

Sekcja źródeł umożliwia użytkownikowi weryfikację poprawności 
odpowiedzi oraz przeczytanie szerszego kontekstu z oryginalnego 
dokumentu.

\newpage
\subsection{Historia konwersacji}

System zapisuje wszystkie konwersacje użytkownika w bazie danych. 
Panel boczny wyświetla listę poprzednich rozmów. Użytkownik ma możliwość
przełączania się między nimi. Rysunek~\ref{fig:demo_history} 
przedstawia interfejs z kilkoma zapisanymi konwersacjami.

\begin{figure}[H]
\centering
\includegraphics[width=0.9\textwidth]{images/implementation/demo_history.png}
\caption{Panel boczny z historią konwersacji}
\label{fig:demo_history}
\end{figure}

Każda pozycja na liście zawiera tytuł konwersacji (generowany na podstawie nazwy przesłanego dokumentu) oraz liczbę wiadomości. Użytkownik może również usunąć wybraną konwersację klikając przycisk ,,×''.

\subsection{Wznawianie konwersacji}

Ważną funkcjonalnością systemu jest możliwość wznawiania 
poprzednich rozmów. Po kliknięciu użytkownika na konwersację w panelu 
bocznym system ładuje zapisaną bazę wektorową z dysku oraz 
wyświetla pełną historię wiadomości.

Dzięki temu użytkownik może kontynuować pracę z dokumentem 
bez konieczności ponownego przesyłania pliku PDF i oczekiwania 
na przetworzenie. Stan konwersacji jest zachowywany po 
wylogowaniu i ponownym zalogowaniu.