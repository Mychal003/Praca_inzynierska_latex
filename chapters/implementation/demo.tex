W tej sekcji przedstawiono demonstrację działania systemu na 
przykładzie dokumentacji routera TP-Link Archer C7.

\subsection{Logowanie do systemu}

\begin{figure}[H]
\centering
\includegraphics[width=0.7\textwidth]{images/implementation/login_screen.png}
\caption{Ekran logowania do systemu}
\label{fig:demo_login}
\end{figure}

\subsection{Tworzenie konwersacji i przesyłanie dokumentu}

Po uruchomieniu aplikacji użytkownik widzi ekran logowania (Rysunek~\ref{fig:demo_login}). Nowi użytkownicy mogą założyć konto przechodząc do zakładki 
,,Rejestracja''. Po zalogowaniu system przekierowuje do głównego 
interfejsu aplikacji.

\newpage

\begin{figure}[H]
\centering
\includegraphics[width=0.9\textwidth]{images/implementation/po_zalgowaniu.png}
\caption{Interfejs chatu z historią konwersacji}
\label{fig:chat_screen}
\end{figure}

Po zalogowaniu użytkownik widzi panel boczny z listą swoich 
konwersacji~\ref{fig:chat_screen}. Aby rozpocząć nową rozmowę, należy kliknąć przycisk ,,+'' w panelu bocznym, a następnie przesłać dokument PDF.

Proces przetwarzania dokumentu obejmuje ekstrakcję tekstu, 
podział na chunki i generowanie embeddingów. Czas 
przetwarzania dla typowej dokumentacji technicznej wynosi \\do kilku sekund.


\subsection{Zadawanie pytań}

Po przetworzeniu dokumentu użytkownik może zadawać pytania 
w języku naturalnym. System automatycznie klasyfikuje typ 
pytania i dostosowuje format odpowiedzi:

\begin{itemize}
    \item \textbf{Pytania faktyczne} -- zwięzła, konkretna informacja
    \item \textbf{Pytania proceduralne} -- numerowana lista kroków
    \item \textbf{Pytania diagnostyczne} -- diagnoza i propozycje rozwiązań
\end{itemize}
\newpage

\begin{figure}[H]
    \centering
    \includegraphics[width=0.9\textwidth]{images/implementation/demo_factual.png}
    \caption{Przykład odpowiedzi na pytanie faktyczne}
    \label{fig:demo_factual}
\end{figure}

Rysunek~\ref{fig:demo_factual} przedstawia przykład odpowiedzi
na pytanie faktyczne dotyczące specyfikacji urządzenia. Treść pytania i wygenerowanej odpowiedzi to:
\begin{itemize}
    \item \bfseries Pytanie (PL): \normalfont Jaka jest pełna nazwa modelu i typ tego routera?
    \item \bfseries Odpowiedź (PL): \normalfont Pełna nazwa modelu i typ tego routera to TP-LINK Archer D7 AC1750 Wireless Dual Band Gigabit ADSL2+ Modem Router.
\end{itemize}
\newpage
\begin{figure}[H]
    \centering
    \includegraphics[width=0.9\textwidth]{images/implementation/demo_procedural.png}
    \caption{Przykład odpowiedzi na pytanie proceduralne}
    \label{fig:demo_procedural}
\end{figure}

Rysunek~\ref{fig:demo_procedural} przedstawia odpowiedź na pytanie
proceduralne z instrukcją krok po kroku. Treść pytania i wygenerowanej odpowiedzi to:
\begin{itemize}
    \item \bfseries Pytanie: \normalfont Jak włączyć filtrowanie MAC w celu kontrolowania dostępu bezprzewodowego?
    \item \bfseries Odpowiedź: \normalfont 1. Odwiedź http://tplinkmodem.net i zaloguj się za pomocą hasła ustawionego dla routera. 2. Przejdź do Zaawansowane > Bezprzewodowe > Filtrowanie MAC i włącz Filtrowanie MAC sieci bezprzewodowej. 3. Wybierz regułę filtrowania, aby albo zablokować, albo zezwolić urządzeniu(om) z listy. 4. W przypadku blokowania, wybierz „Zablokuj dostęp bezprzewodowy urządzeniom na poniższej liście” i kliknij Zapisz. 5. Wybierz urządzenie(a) do zablokowania w tabeli Urządzenia Online i kliknij Zablokuj. 6. W przypadku zezwalania, wybierz „Zezwól na dostęp bezprzewodowy tylko urządzeniom na poniższej liście” i kliknij Zapisz, a następnie kliknij Dodaj, \\aby wprowadzić adres MAC i opis urządzenia, i kliknij OK.
\end{itemize}



\newpage
\subsection{Wyświetlanie źródeł}


\begin{figure}[H]
\centering
\includegraphics[width=0.9\textwidth]{images/implementation/demo_sources1.png}
\caption{Źródła użyte do wygenerowania odpowiedzi}
\label{fig:demo_sources}
\end{figure}
Każda odpowiedź zawiera rozwijalną sekcję ,,Źródła'' -- fragmenty 
dokumentu użyte \\do wygenerowania odpowiedzi. 
Rysunek~\ref{fig:demo_sources} przedstawia przykład wyświetlonych źródeł.

Sekcja źródeł umożliwia użytkownikowi weryfikację poprawności 
odpowiedzi oraz przeczytanie szerszego kontekstu z oryginalnego 
dokumentu w razie wątpliwości co do otrzymanej odpowiedzi.

\newpage
\subsection{Historia konwersacji}



\begin{figure}[H]
\centering
\includegraphics[width=0.9\textwidth]{images/implementation/demo_history1.png}
\caption{Panel boczny z historią konwersacji}
\label{fig:demo_history}
\end{figure}
System zapisuje wszystkie konwersacje użytkownika w bazie danych. 
Panel boczny wyświetla listę poprzednich rozmów. Użytkownik ma możliwość
przełączania się między nimi. Rysunek~\ref{fig:demo_history} 
przedstawia interfejs z kilkoma zapisanymi konwersacjami.
Każda pozycja na liście zawiera tytuł konwersacji (generowany na podstawie nazwy przesłanego dokumentu) oraz liczbę wiadomości. Użytkownik może również usunąć wybraną konwersację klikając przycisk ,,×''.

\subsection{Wznawianie konwersacji}

Ważną funkcjonalnością systemu jest możliwość wznawiania 
poprzednich rozmów. \\Po kliknięciu użytkownika na konwersację w panelu 
bocznym system ładuje zapisaną bazę wektorową z dysku oraz 
wyświetla pełną historię wiadomości.

Dzięki temu użytkownik może kontynuować pracę z dokumentem 
bez konieczności ponownego przesyłania pliku PDF i oczekiwania 
na przetworzenie. Stan konwersacji jest zachowywany \\po 
wylogowaniu i ponownym zalogowaniu.