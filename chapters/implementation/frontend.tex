Frontend systemu został zaimplementowany jako aplikacja webowa 
wykorzystująca HTML5, CSS3 oraz JavaScript. Interfejs zaprojektowano 
z myślą o intuicyjności, czytelności odpowiedzi oraz możliwości 
zarządzania wieloma konwersacjami.

\subsection{Struktura frontendu}
\begin{figure}[H]
    \centering
    \includegraphics[width=0.3\textwidth]{images/implementation/front_dir.png}
    \caption{Struktura plików frontend}
    \label{fig:frontend_structure}
\end{figure}

Rysunek Rysunek~\ref{fig:frontend_structure} przedstawia główne pliki aplikacji:
\begin{itemize}
    \item \texttt{index.html} -- struktura HTML z ekranami logowania i chatu
    \item \texttt{auth.js} -- obsługa autoryzacji (JWT, logowanie, rejestracja)
    \item \texttt{app.js} -- logika chatu i zarządzanie konwersacjami
    \item \texttt{style.css} -- style CSS z ciemnym motywem
\end{itemize}




\subsection{Interfejs użytkownika}

Interfejs składa się z trzech głównych ekranów:

\begin{enumerate}
    \item \textbf{Ekran logowania} -- formularz logowania i rejestracji użytkownika
    \item \textbf{Sekcja uploadu} -- wybór i przesłanie pliku PDF do przetworzenia
    \item \textbf{Sekcja chatu} -- historia konwersacji, sidebar z listą rozmów 
    oraz pole do wpisywania pytań
\end{enumerate}
\newpage


\begin{figure}[H]
\centering
\includegraphics[width=0.7\textwidth]{images/implementation/login_screen.png}
\caption{Ekran logowania}
\label{fig:login_screen}
\end{figure}

\begin{figure}[H]
\centering
\includegraphics[width=0.7\textwidth]{images/implementation/Rejestr.png}
\caption{Ekran rejestracji}
\label{fig:register_screen}
\end{figure}

Rysunki~\ref{fig:login_screen} i~\ref{fig:register_screen} przedstawiają ekran logowania oraz rejestracji.

\newpage


\begin{figure}[H]
\centering
\includegraphics[width=0.9\textwidth]{images/implementation/upload_screen1.png}
\caption{Ekran uploadu dokumentu PDF z panelem konwersacji}
\label{fig:upload_screen}
\end{figure}

Po zalogowaniu użytkownik widzi interfejs z bocznym panelem zawierającym 
listę swoich konwersacji. Rysunek~\ref{fig:upload_screen} przedstawia 
ekran uploadu dokumentu.



\begin{figure}[H]
\centering
\includegraphics[width=0.9\textwidth]{images/implementation/chat_screen1.png}
\caption{Interfejs chatu z historią konwersacji}
\label{fig:chat_screen}
\end{figure}

Po przesłaniu dokumentu użytkownik przechodzi do interfejsu chatu 
przedstawionego na rysunku~\ref{fig:chat_screen}. Panel boczny 
umożliwia przełączanie między konwersacjami oraz tworzenie nowych.

\newpage
\subsection{System autoryzacji}

Moduł \texttt{auth.js} zarządza tokenami JWT i stanem użytkownika. 
Listing~\ref{lst:auth_module} przedstawia główne funkcje autoryzacji.
Poniższy listing przedstawia główne funkcje modułu \texttt{auth.js}:

\begin{itemize}
    \item \textbf{getToken()} -- pobiera aktualny token JWT z pamięci lokalnej
    \item \textbf{setToken(token)} -- zapisuje token JWT w pamięci lokalnej
    \item \textbf{getHeaders()} -- przygotowuje nagłówki HTTP z tokenem dla żądań
    \item \textbf{login(username, password)} -- wysyła dane logowania do API,
    odbiera token JWT \\i zapisuje go w lokalnej pamięci
\end{itemize}


\begin{lstlisting}[language=JavaScript, caption={Moduł autoryzacji}, label={lst:auth_module}]
const Auth = {
    getToken() {
        return localStorage.getItem('auth_token');
    },
    
    setToken(token) {
        localStorage.setItem('auth_token', token);
    },
    
    getHeaders() {
        const headers = {'Content-Type': 'application/json'};
        const token = this.getToken();
        if (token) {
            headers['Authorization'] = `Bearer ${token}`;
        }
        return headers;
    },
    
    async login(username, password) {
        const response = await fetch(`${API_URL}/auth/login`, {
            method: 'POST',
            headers: {'Content-Type': 'application/json'},
            body: JSON.stringify({username, password})
        });
        const data = await response.json();
        if (response.ok) {
            this.setToken(data.token);
        }
        return data;
    }
};
\end{lstlisting}

\newpage
\subsection{Zarządzanie konwersacjami}

Aplikacja umożliwia użytkownikowi tworzenie nowych konwersacji,
przeglądanie historii oraz ponowne otwieranie wcześniejszych rozmów.
Listing~\ref{lst:conversations} przedstawia przedstawia główne funkcje odpowiedzialne za
zarządzanie konwersacjami:

\begin{itemize}
    \item \textbf{loadConversations()} -- pobiera listę rozmów użytkownika
    z API i aktualizuje interfejs
    \item \textbf{createNewConversation()} -- tworzy nową konwersację
    w bazie danych i ustawia \\ją jako aktywną
    \item \textbf{openConversation(conversationId)} -- pobiera treść wybranej
    konwersacji oraz wyświetla jej wiadomości w interfejsie chatu
\end{itemize}

\begin{lstlisting}[language=JavaScript, caption={Zarządzanie konwersacjami}, label={lst:conversations}]
async function loadConversations() {
    const response = await apiCall('/conversations');
    const data = await response.json();
    renderConversationsList(data.conversations);
}

async function createNewConversation() {
    const response = await apiCall('/conversations', {
        method: 'POST'
    });
    const data = await response.json();
    currentConversationId = data.conversation.id;
    await loadConversations();
}

async function openConversation(conversationId) {
    const response = await apiCall(
        `/conversations/${conversationId}`
    );
    const data = await response.json();
    currentConversationId = conversationId;
    renderMessages(data.conversation.messages);
}
\end{lstlisting}


\newpage
\subsection{Komunikacja z API}

Komunikacja aplikacji z backendem odbywa się za pomocą funkcji \texttt{apiCall}, która dodaje wymagane nagłówki (m.in. autoryzacyjne) i stndaryzuje obsługę błędów. Podczas wysyłania pytania wiadomość użytkownika jest dodawana do interfejsu, a następnie zapytanie trafia do systemu RAG.

Odpowiedź z API zawiera:
\begin{itemize}
\item wygenerowaną odpowiedź asystenta,
\item kategorię pytania,
\item listę fragmentów źródłowych z dokumentacji.
\end{itemize}

Listing~\ref{lst:ask_question} przedstawia funkcję odpowiedzialną za przesłanie pytania i wyświetlenie odpowiedzi po stronie klienta.

\begin{lstlisting}[language=JavaScript, caption={Wysyłanie pytania do API}, label={lst:ask_question}]
async function askQuestion() {
    const input = document.getElementById('questionInput');
    const question = input.value.trim();
    if (!question) return;
    
    addMessageToUI(question, 'user');
    input.value = '';
    
    try {
        const response = await apiCall(
            `/conversations/${currentConversationId}/query`, 
            {
                method: 'POST',
                body: JSON.stringify({question})
            }
        );
        const data = await response.json();
        
        if (response.ok) {
            addMessageToUI(data.answer, 'assistant', 
                          data.category, data.sources);
        } else {
            addMessageToUI(`Blad: ${data.error}`, 'error');
        }
    } catch (error) {
        addMessageToUI(`Blad: ${error.message}`, 'error');
    }
}
\end{lstlisting}


\newpage