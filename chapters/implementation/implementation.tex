\chapter{Implementacja}\label{chapter:implementation}

Rozdział opisuje szczegóły implementacji systemu RAG do obsługi 
dokumentacji technicznej. System składa się z backendu napisanego 
w Pythonie z użyciem frameworka Flask oraz frontendu w postaci 
aplikacji webowej. 

\begin{figure}[H]
\centering
\includegraphics[width=1.0\textwidth]{images/implementation/final1.drawio.png}
\caption{Architektura systemu RAG}
\label{fig:architecture}
\end{figure}
Rysunek~\ref{fig:architecture} przedstawia 
ogólną architekturę systemu.\newline
System został podzielony na trzy główne warstwy:
\begin{itemize}
    \item \textbf{Warstwa prezentacji} -- interfejs webowy umożliwiający 
    upload dokumentów i zadawanie pytań
    \item \textbf{Warstwa API} -- serwer Flask obsługujący żądania HTTP
    \item \textbf{Warstwa przetwarzania} -- pipeline RAG z bazą wektorową FAISS
\end{itemize}
\newpage
\section{Backend}\label{sec:backend}
Backend systemu został zaimplementowany w języku Python z wykorzystaniem 
frameworka Flask. Składa się z kilku głównych modułów: API HTTP, 
system autoryzacji, modele bazy danych, pipeline RAG oraz klasyfikator pytań.

\subsection{Struktura projektu}



\begin{figure}[h!]
    \centering
    \includegraphics[width=0.35\textwidth]{images/implementation/backend_dir.png}
    \caption{Struktura katalogów backendu}
    \label{fig:backend_structure}
\end{figure}
Struktura katalogów backendu została przedstawiona jest na rysunku~\ref{fig:backend_structure}.
Główne pliki aplikacji:
\begin{itemize}
    \item \texttt{app.py} -- główny plik serwera Flask z endpointami API
    \item \texttt{config.py} -- konfiguracja aplikacji (klucze, ścieżki)
    \item \texttt{models.py} -- modele bazy danych SQLAlchemy
    \item \texttt{auth.py} -- obsługa autoryzacji JWT
    \item \texttt{conversation\_manager.py} -- zarządzanie historią rozmów
\end{itemize}
\newpage
\subsection{API HTTP}

Serwer Flask udostępnia endpointy REST API podzielone na trzy grupy 
funkcjonalne.
\texttt{Authorization: Bearer <token>}.
\begin{table}[H]
\centering
\caption{Endpointy API}
\label{tab:endpoints}
\begin{tabular}{|l|l|p{5.5cm}|}
\hline
\textbf{Metoda} & \textbf{Endpoint} & \textbf{Opis} \\
\hline
\multicolumn{3}{|c|}{\textit{Autoryzacja}} \\
\hline
POST & /api/auth/register & Rejestracja nowego użytkownika \\
\hline
POST & /api/auth/login & Logowanie i generowanie tokenu JWT \\
\hline
GET & /api/auth/me & Pobranie danych zalogowanego użytkownika \\
\hline
\multicolumn{3}{|c|}{\textit{Konwersacje}} \\
\hline
GET & /api/conversations & Lista konwersacji użytkownika \\
\hline
POST & /api/conversations & Utworzenie nowej konwersacji \\
\hline
GET & /api/conversations/<id> & Szczegóły konwersacji z wiadomościami \\
\hline
DELETE & /api/conversations/<id> & Usunięcie konwersacji \\
\hline
\multicolumn{3}{|c|}{\textit{Dokumenty i zapytania}} \\
\hline
POST & /api/conversations/<id>/upload & Upload dokumentu PDF \\
\hline
POST & /api/conversations/<id>/query & Zadanie pytania do dokumentu \\
\hline
GET & /api/health & Sprawdzenie statusu serwera \\
\hline
\end{tabular}
\end{table}

 Tabela~\ref{tab:endpoints} przedstawia pełną listę 
dostępnych endpointów.
Wszystkie endpointy z grup \textit{Konwersacje} oraz \textit{Dokumenty i zapytania} 
wymagają autoryzacji za pomocą tokenu JWT przekazywanego w nagłówku 

\newpage
\subsection{Modele bazy danych}

System wykorzystuje bazę SQLite z trzema głównymi modelami. 
Listing~\ref{lst:models} przedstawia definicje modeli.

Modele bazy danych wykorzystują relacje typu jeden–do–wielu. 
Model \texttt{User} reprezentuje użytkownika systemu i przechowuje jego
dane logowania oraz listę prowadzonych rozmów. Każdy użytkownik może
mieć wiele rozmów.

Model \texttt{Conversation} definiuje pojedynczą rozmowę związaną z
konkretnym użytkownikiem. Oprócz tytułu zawiera informację o pliku
dokumentu, na podstawie którego generowane są odpowiedzi systemu.

Model \texttt{Message} przechowuje wiadomości należące do danej rozmowy. Dodatkowymi polami są
kategoria pytania oraz źródła informacji wykorzystane przy generowaniu
odpowiedzi przez model językowy.


\begin{lstlisting}[language=Python, caption={Modele bazy danych}, label={lst:models}]
class User(db.Model):
    id = db.Column(db.Integer, primary_key=True)
    username = db.Column(db.String(80), unique=True, nullable=False)
    email = db.Column(db.String(120), unique=True, nullable=False)
    password_hash = db.Column(db.String(256), nullable=False)
    created_at = db.Column(db.DateTime, default=datetime.utcnow)
    
    conversations = db.relationship('Conversation', backref='user')

class Conversation(db.Model):
    id = db.Column(db.Integer, primary_key=True)
    user_id = db.Column(db.Integer, db.ForeignKey('users.id'))
    title = db.Column(db.String(200), default='Nowa rozmowa')
    document_name = db.Column(db.String(200))
    created_at = db.Column(db.DateTime, default=datetime.utcnow)
    
    messages = db.relationship('Message', backref='conversation')

class Message(db.Model):
    id = db.Column(db.Integer, primary_key=True)
    conversation_id = db.Column(db.Integer, db.ForeignKey('conversations.id'))
    role = db.Column(db.String(20))
    content = db.Column(db.Text, nullable=False)
    category = db.Column(db.String(50))
    sources = db.Column(db.JSON)
    created_at = db.Column(db.DateTime, default=datetime.utcnow)
\end{lstlisting}

\newpage
\subsection{System autoryzacji}

System autoryzacji opiera się na tokenach JWT (JSON Web Token), które
pozwalają \\na uwierzytelnianie użytkowników bez konieczności
ponownego logowania, token traci ważność po 24 godzinach. Przechowuje on również
identyfikator użytkownika. Po wysłaniu żądania token jest przekazywany w nagłówku HTTP \texttt{Authorization}.

Listing~\ref{lst:auth} przedstawia dekorator \texttt{@token\_required},
który weryfikuje poprawność tokenu przed wykonaniem chronionych
endpointów.
\begin{lstlisting}[language=Python, caption={Dekorator autoryzacji JWT}, label={lst:auth}]
def token_required(f):
    @wraps(f)
    def decorated(*args, **kwargs):
        token = None
        
        if 'Authorization' in request.headers:
            auth_header = request.headers['Authorization']
            if auth_header.startswith('Bearer '):
                token = auth_header.split(' ')[1]
        
        if not token:
            return jsonify({'error': 'Brak tokenu'}), 401
        
        user_id = decode_token(token, current_app.config['SECRET_KEY'])
        if not user_id:
            return jsonify({'error': 'Token nieprawidlowy'}), 401
        
        user = User.query.get(user_id)
        return f(user, *args, **kwargs)
    
    return decorated
\end{lstlisting}


\newpage
\subsection{Pipeline RAG}

Główna klasa \texttt{RAGPipeline} implementuje proces przetwarzania 
dokumentu i generowania odpowiedzi. Listing~\ref{lst:rag_init} przedstawia 
inicjalizację pipeline z konfigurowalnymi parametrami.
Metoda \texttt{process\_document} przetwarza dokument PDF w czterech krokach:

\begin{enumerate}
    \item \textbf{Ekstrakcja tekstu} -- biblioteka PyMuPDF wyodrębnia 
    tekst z pliku PDF
    \item \textbf{Chunking} -- tekst jest dzielony na fragmenty o rozmiarze 
    800 tokenów z nakładką 100 tokenów
    \item \textbf{Embeddingi} -- każdy fragment jest przekształcany 
    w wektor przy użyciu modelu \texttt{text-embedding-3-large}
    \item \textbf{Indeksowanie} -- wektory są zapisywane w bazie FAISS
\end{enumerate}

Baza wektorowa (vector store) jest zapisywana na dysku w katalogu \texttt{vectorstores/}, 
\\co umożliwia wznawianie konwersacji bez ponownego przetwarzania dokumentu.


\begin{lstlisting}[language=Python, caption={Inicjalizacja RAG Pipeline}, label={lst:rag_init}]
class RAGPipeline:
    def __init__(self, chunk_size=800, chunk_overlap=100, k=10):
        self.chunk_size = chunk_size
        self.chunk_overlap = chunk_overlap
        self.k = k
        
        self.pdf_processor = PDFProcessor()
        self.classifier = QueryClassifier()
        self.embeddings = OpenAIEmbeddings(
            model="text-embedding-3-large"
        )
        self.llm = ChatOpenAI(model="gpt-4o", temperature=0)
        self.vectorstore = None
\end{lstlisting}


\newpage
\subsection{Klasyfikator pytań}

System wykorzystuje klasyfikator oparty na modelu GPT-4o do 
kategoryzacji pytań użytkownika na trzy typy:

\begin{itemize}
    \item \textbf{factual} -- pytania o konkretne fakty i wartości
    \item \textbf{procedural} -- pytania o instrukcje i procedury
    \item \textbf{troubleshooting} -- pytania o rozwiązywanie problemów
\end{itemize}

\begin{lstlisting}[language=Python, caption={Klasyfikator pytań}, label={lst:classifier}]
class QueryClassifier:
    CATEGORIES = {
        "factual": "Pytanie o fakt",
        "procedural": "Pytanie o instrukcje",
        "troubleshooting": "Pytanie o problem"
    }
    
    def classify(self, query: str) -> str:
        prompt = f"""Sklasyfikuj pytanie do kategorii:
        - factual: pytanie o fakt
        - procedural: pytanie o instrukcje
        - troubleshooting: pytanie o problem
        
        Pytanie: "{query}"
        Odpowiedz TYLKO nazwa kategorii."""
        
        response = self.llm.invoke(prompt)
        result = response.content.strip().lower()
        return result if result in self.CATEGORIES else "factual"
\end{lstlisting}
\newpage
\subsection{Generowanie odpowiedzi}

W zależności od kategorii pytania wybierany jest odpowiedni szablon 
promptu zoptymalizowany pod dany typ zapytania. Listing~\ref{lst:query} 
przedstawia metodę generowania odpowiedzi.
Odpowiedź wraz z kategorią i źródłami jest zapisywana w bazie danych 
jako nowa wiadomość w ramach konwersacji.

\begin{lstlisting}[language=Python, caption={Generowanie odpowiedzi}, label={lst:query}]
def query(self, question: str) -> str:
    # 1. Klasyfikuj pytanie
    category = self.classify_query(question)
    
    # 2. Pobierz odpowiedni prompt
    prompt_template = self._get_prompt_template(category)
    
    # 3. Stworz QA chain z retrieverem
    qa_chain = RetrievalQA.from_chain_type(
        llm=self.llm,
        chain_type="stuff",
        retriever=self.retriever,
        return_source_documents=True,
        chain_type_kwargs={"prompt": prompt_template}
    )
    
    # 4. Wygeneruj odpowiedz
    result = qa_chain({"query": question})
    return result["result"]
\end{lstlisting}


\newpage
\section{Frontend}\label{sec:frontend}
Frontend systemu został zaimplementowany jako aplikacja webowa 
wykorzystująca HTML5, CSS3 oraz JavaScript. Interfejs zaprojektowano 
z myślą o intuicyjności, czytelności odpowiedzi oraz możliwości 
zarządzania wieloma konwersacjami.

\subsection{Struktura frontendu}

\dirtree{%
.1 frontend/.
.2 index.html.
.2 static/.
.3 css/.
.4 style.css.
.3 js/.
.4 auth.js.
.4 app.js.
}

Główne pliki aplikacji:
\begin{itemize}
    \item \texttt{index.html} -- struktura HTML z ekranami logowania i chatu
    \item \texttt{auth.js} -- obsługa autoryzacji (JWT, logowanie, rejestracja)
    \item \texttt{app.js} -- logika chatu i zarządzanie konwersacjami
    \item \texttt{style.css} -- style CSS z ciemnym motywem
\end{itemize}

\subsection{Interfejs użytkownika}

Interfejs składa się z trzech głównych ekranów:

\begin{enumerate}
    \item \textbf{Ekran logowania} -- formularz logowania i rejestracji użytkownika
    \item \textbf{Sekcja uploadu} -- wybór i przesłanie pliku PDF do przetworzenia
    \item \textbf{Sekcja chatu} -- historia konwersacji, sidebar z listą rozmów 
    oraz pole do wpisywania pytań
\end{enumerate}
\newpage
Rysunki~\ref{fig:login_screen} i~\ref{fig:register_screen} przedstawiają ekran logowania oraz rejestracji.

\begin{figure}[H]
\centering
\includegraphics[width=0.7\textwidth]{images/implementation/login_screen.png}
\caption{Ekran logowania}
\label{fig:login_screen}
\end{figure}

\begin{figure}[H]
\centering
\includegraphics[width=0.7\textwidth]{images/implementation/Rejestr.png}
\caption{Ekran rejestracji}
\label{fig:register_screen}
\end{figure}


\newpage
Po zalogowaniu użytkownik widzi interfejs z bocznym panelem zawierającym 
listę swoich konwersacji. Rysunek~\ref{fig:upload_screen} przedstawia 
ekran uploadu dokumentu.

\begin{figure}[H]
\centering
\includegraphics[width=0.9\textwidth]{images/implementation/upload_screen1.png}
\caption{Ekran uploadu dokumentu PDF z panelem konwersacji}
\label{fig:upload_screen}
\end{figure}


Po przesłaniu dokumentu użytkownik przechodzi do interfejsu chatu 
przedstawionego na rysunku~\ref{fig:chat_screen}. Panel boczny 
umożliwia przełączanie między konwersacjami oraz tworzenie nowych.

\begin{figure}[H]
\centering
\includegraphics[width=0.9\textwidth]{images/implementation/chat_screen1.png}
\caption{Interfejs chatu z historią konwersacji}
\label{fig:chat_screen}
\end{figure}

\newpage
\subsection{System autoryzacji}

Moduł \texttt{auth.js} zarządza tokenami JWT i stanem użytkownika. 
Listing~\ref{lst:auth_module} przedstawia główne funkcje autoryzacji.

\begin{lstlisting}[language=JavaScript, caption={Moduł autoryzacji}, label={lst:auth_module}]
const Auth = {
    getToken() {
        return localStorage.getItem('auth_token');
    },
    
    setToken(token) {
        localStorage.setItem('auth_token', token);
    },
    
    getHeaders() {
        const headers = {'Content-Type': 'application/json'};
        const token = this.getToken();
        if (token) {
            headers['Authorization'] = `Bearer ${token}`;
        }
        return headers;
    },
    
    async login(username, password) {
        const response = await fetch(`${API_URL}/auth/login`, {
            method: 'POST',
            headers: {'Content-Type': 'application/json'},
            body: JSON.stringify({username, password})
        });
        const data = await response.json();
        if (response.ok) {
            this.setToken(data.token);
        }
        return data;
    }
};
\end{lstlisting}

\newpage
\subsection{Zarządzanie konwersacjami}

Aplikacja umożliwia tworzenie, przeglądanie i usuwanie konwersacji. 
Listing~\ref{lst:conversations} przedstawia funkcje zarządzania rozmowami.

\begin{lstlisting}[language=JavaScript, caption={Zarządzanie konwersacjami}, label={lst:conversations}]
async function loadConversations() {
    const response = await apiCall('/conversations');
    const data = await response.json();
    renderConversationsList(data.conversations);
}

async function createNewConversation() {
    const response = await apiCall('/conversations', {
        method: 'POST'
    });
    const data = await response.json();
    currentConversationId = data.conversation.id;
    await loadConversations();
}

async function openConversation(conversationId) {
    const response = await apiCall(
        `/conversations/${conversationId}`
    );
    const data = await response.json();
    currentConversationId = conversationId;
    renderMessages(data.conversation.messages);
}
\end{lstlisting}

\newpage
\subsection{Komunikacja z API}

Wszystkie żądania do API są realizowane przez funkcję pomocniczą 
\texttt{apiCall}, która automatycznie dodaje nagłówki autoryzacji. 
Listing~\ref{lst:ask_question} przedstawia funkcję wysyłania pytania.

\begin{lstlisting}[language=JavaScript, caption={Wysyłanie pytania do API}, label={lst:ask_question}]
async function askQuestion() {
    const input = document.getElementById('questionInput');
    const question = input.value.trim();
    if (!question) return;
    
    addMessageToUI(question, 'user');
    input.value = '';
    
    try {
        const response = await apiCall(
            `/conversations/${currentConversationId}/query`, 
            {
                method: 'POST',
                body: JSON.stringify({question})
            }
        );
        const data = await response.json();
        
        if (response.ok) {
            addMessageToUI(data.answer, 'assistant', 
                          data.category, data.sources);
        } else {
            addMessageToUI(`Blad: ${data.error}`, 'error');
        }
    } catch (error) {
        addMessageToUI(`Blad: ${error.message}`, 'error');
    }
}
\end{lstlisting}

Odpowiedzi asystenta zawierają kategorię pytania oraz źródła 
(fragmenty dokumentu), które można rozwinąć klikając w sekcję 
,,Źródła'' pod odpowiedzią.
\newpage

\section{Demonstracja działania}\label{sec:demo}
W tej sekcji przedstawiono demonstrację działania systemu na 
przykładzie dokumentacji routera TP-Link Archer C7.

\subsection{Logowanie do systemu}

Po uruchomieniu aplikacji użytkownik widzi ekran logowania (Rysunek~\ref{fig:demo_login}). Nowi użytkownicy mogą założyć konto przechodząc do zakładki 
,,Rejestracja''. Po zalogowaniu system przekierowuje do głównego 
interfejsu aplikacji.

\begin{figure}[H]
\centering
\includegraphics[width=0.7\textwidth]{images/implementation/login_screen.png}
\caption{Ekran logowania do systemu}
\label{fig:demo_login}
\end{figure}

\subsection{Tworzenie konwersacji i przesyłanie dokumentu}

Po zalogowaniu użytkownik widzi panel boczny z listą swoich 
konwersacji. Aby rozpocząć nową rozmowę, należy kliknąć przycisk 
,,+'' w panelu bocznym, a następnie przesłać dokument PDF.

Proces przetwarzania dokumentu obejmuje ekstrakcję tekstu, 
podział na chunki oraz generowanie embeddingów. Czas 
przetwarzania dla typowej dokumentacji technicznej wynosi kilka sekund.

\newpage
\subsection{Zadawanie pytań}

Po przetworzeniu dokumentu użytkownik może zadawać pytania 
w języku naturalnym. System automatycznie klasyfikuje typ 
pytania i dostosowuje format odpowiedzi:

\begin{itemize}
    \item \textbf{Pytania faktyczne} -- zwięzła, konkretna informacja
    \item \textbf{Pytania proceduralne} -- numerowana lista kroków
    \item \textbf{Pytania o problemy} -- diagnoza i propozycje rozwiązań
\end{itemize}

Rysunek~\ref{fig:demo_factual} przedstawia przykład odpowiedzi 
na pytanie faktyczne dotyczące specyfikacji urządzenia.

\begin{figure}[H]
\centering
\includegraphics[width=0.9\textwidth]{images/implementation/demo_factual.png}
\caption{Przykład odpowiedzi na pytanie faktyczne}
\label{fig:demo_factual}
\end{figure}

\newpage
Rysunek~\ref{fig:demo_procedural} przedstawia odpowiedź na pytanie 
proceduralne z instrukcją krok po kroku.

\begin{figure}[H]
\centering
\includegraphics[width=0.9\textwidth]{images/implementation/demo_procedural.png}
\caption{Przykład odpowiedzi na pytanie proceduralne}
\label{fig:demo_procedural}
\end{figure}

\newpage
\subsection{Wyświetlanie źródeł}

Każda odpowiedź zawiera rozwijalną sekcję ,,Źródła'' -- fragmenty 
dokumentu użyte do wygenerowania odpowiedzi. 
Rysunek~\ref{fig:demo_sources} przedstawia przykład wyświetlonych źródeł.

\begin{figure}[H]
\centering
\includegraphics[width=0.9\textwidth]{images/implementation/demo_sources1.png}
\caption{Źródła użyte do wygenerowania odpowiedzi}
\label{fig:demo_sources}
\end{figure}

Sekcja źródeł umożliwia użytkownikowi weryfikację poprawności 
odpowiedzi oraz przeczytanie szerszego kontekstu z oryginalnego 
dokumentu.

\newpage
\subsection{Historia konwersacji}

System zapisuje wszystkie konwersacje użytkownika w bazie danych. 
Panel boczny wyświetla listę poprzednich rozmów. Użytkownik ma możliwość
przełączania się między nimi. Rysunek~\ref{fig:demo_history} 
przedstawia interfejs z kilkoma zapisanymi konwersacjami.

\begin{figure}[H]
\centering
\includegraphics[width=0.9\textwidth]{images/implementation/demo_history.png}
\caption{Panel boczny z historią konwersacji}
\label{fig:demo_history}
\end{figure}

Każda pozycja na liście zawiera tytuł konwersacji (generowany na podstawie nazwy przesłanego dokumentu) oraz liczbę wiadomości. Użytkownik może również usunąć wybraną konwersację klikając przycisk ,,×''.

\subsection{Wznawianie konwersacji}

Ważną funkcjonalnością systemu jest możliwość wznawiania 
poprzednich rozmów. Po kliknięciu użytkownika na konwersację w panelu 
bocznym system ładuje zapisaną bazę wektorową z dysku oraz 
wyświetla pełną historię wiadomości.

Dzięki temu użytkownik może kontynuować pracę z dokumentem 
bez konieczności ponownego przesyłania pliku PDF i oczekiwania 
na przetworzenie. Stan konwersacji jest zachowywany po 
wylogowaniu i ponownym zalogowaniu.