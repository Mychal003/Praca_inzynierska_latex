\chapter{Wprowadzenie}\label{chapter:introduction}

Szybki postęp w dziedzinie  sztucznej inteligencji, a w szczególności w tworzeniu dużych modeli językowych (LLM -- Large Language Models), otworzył nowe możliwości w zakresie przetwarzania i analizy tekstu. Współaczesne modele -- np GPT-4 potrafią generować odpowiedzi
na pytania użytkowników w sposób przypominający konwersacje z drugim człowiekiem. Jednakże same modele językowe
mają istotne ograniczenie -- ich wiedza pochodzi wyłącznie z danych treningowych, oznacza to, że 
 nie mają one dostępu do wewnętrzych dokumentów firmowych czy szcegółowej dokumentacji technicznej. Pomimo, że teoretycznie istnieje możliwość dostosowania modelu do konkretnej
dziedziny poprzez proces finetuningu, to w praktyce jest to  mało opłcalne. Takie rozwiązanie jest bowiem obarczone wieloma wadami, jest kosztowne,
czasochłonne oraz wymagające przygotowania dużych zestawów wysokiej jakości danych. Przy tym nie bez znaczenia jest fakt, że nawet niewielkie błędy w gromadzonych danych mogą zaburzyć i wypaczyć działanie modelu. Pondto, po zakończeniu
finetuningu model staje się „zamknięty” na bieżące aktualizacje -- aby wprowadzić nowe informacje, konieczne jest ponowne przeprowadzenie procesu trenowaniea. Tu rodzi się pytanie o sens stosowania tego podejścia w środowiskach, w któych dokunentacja ulega regularnym zmianą.

To istotne ograniczenie można wyeliminować wykorzystując achitekturę RAG (Retrieval--Augmented Generation) w której proces generowania odpowiedzi jest wspierany mechanizmem wyszukiwania informacji. System odnajduje informacje, które są powiązane z pytaniem użytkownika, a następnie przekazuje je modelowi językowemu jako kontekst. W konsekfencji model może udzielać precyzyjnych odpowiedzi dotyczących dowolnych materiłów źródłowych --  bez potrzeby trenowania czy modyfikowania samego modelu.


Niniejsza praca przedstawia projekt, implementację oraz ewaluację systemu RAG 
przeznaczonego do obsługi dokumentacji technicznej. System został przetestowany 
na instrukcji obsługi routera TP--Link Archer D7, jednakże jego architektura pozwala 
na zastosowanie z dowolnymi dokumentami PDF.

Głównym wkładem pracy jest kompleksowa metodologia ewaluacji systemu RAG oraz 
analiza wpływu różnych parametrów na jakość odpowiedzi. Przeprowadzone eksperymenty 
wykazały, że odpowiednie projektowanie promptów ma kluczowe 
znaczenie dla skuteczności systemu -- większe niż optymalizacja parametrów 
technicznych, takich jak rozmiar fragmentów czy liczba pobieranych dokumentów.
\newpage
\section{Sformułowanie problemu}

Dokumentacja nowoczesnych, zaawansowanych produktów, często liczy dziesiątki lub 
setki stron. Użytkownicy szukający konkretnej informacji muszą przeszukiwać 
dokumenty ręcznie, co jest czasochłonne. Tradycyjne wyszukiwanie 
tekstowe (Ctrl+F) wymaga znajomości dokładnych słów kluczowych i nie radzi sobie 
z pytaniami zadawanymi w języku naturalnym.

\textbf{Problem badawczy:} Jak zbudować system, który automatycznie odpowiada 
na pytania użytkowników na podstawie dokumentacji technicznej, zachowując 
wysoką dokładność i nie generując przy tym nieprawdziwych informacji (halucynacji)?

\textbf{Pytania badawcze:}
\begin{enumerate}
    \item Jak parametry przetwarzania dokumentu (rozmiar fragmentów, nakładanie się, liczba pobieranych dokumentów) wpływają na jakość odpowiedzi?
    \item Jaki wpływ na jakość systemu ma projektowanie promptów w porównaniu 
    z optymalizacją parametrów technicznych?
    \item Które metryki najlepiej oceniają jakość systemu RAG - metryki automatyczne 
    (ROUGE, Semantic Similarity) czy ocena przez model językowy (LLM Judge)?
\end{enumerate}

\textbf{Znaczenie badania:} Systemy RAG znajdują coraz szersze zastosowanie 
w obsłudze klienta, wewnętrznych systemach firmowych i asystentach AI. 
zdiagnozowanie i analiza czynników wpływających na ich jakość pozwala budować lepsze 
rozwiązania i unikać typowych błędów.

\textbf{Ograniczenia badań:}
\begin{itemize}
    \item Ewaluacja przeprowadzona na jednym dokumencie (instrukcja routera, 119 stron)
    \item Dataset ewaluacyjny zawiera 25 pytań z trzech kategorii
    \item System obsługuje tylko dokumenty w formacie PDF
\end{itemize}

\section{Cele pracy}

Głównym celem pracy jest zaprojektowanie i zaimplemantowanie systemu RAG, wraz z jego kompleksową ewaluacją służącego do automatycznego odpowiadania na pytania użytkownika na podstawie dostarczonej dokumentacji technicznej.

\textbf{Cele szczegółowe:}
\begin{enumerate}
    \item \textbf{Implementacja systemu RAG} --- stworzenie działającego systemu 
    obejmującego przetwarzanie PDF, tworzenie embeddingów, bazę wektorową FAISS 
    oraz generowanie odpowiedzi z użyciem modelu GPT-4o.
    
    \item \textbf{Opracowanie metodologii ewaluacji} -- stworzenie zestawu narzędzi 
    do oceny jakości systemu obejmującego:
    \begin{itemize}
        \item metryki wyszukiwania (Precision, Recall, MRR, NDCG)
        \item metryki generacji (ROUGE, Semantic Similarity)
        \item ocenę przez model językowy (LLM Judge)
    \end{itemize}
    
    \item \textbf{Realizacja eksperymentów} -- analiza wpływu parametrów:
    \begin{itemize}
        \item rozmiar fragmentów (chunk size): 300, 500, 800, 1200, 1500 znaków
        \item liczba pobieranych dokumentów (k): 1, 3, 5, 7, 10
        \item nakładanie się fragmentów (overlap): 0, 50, 100, 200, 300 znaków
    \end{itemize}
    
    \item \textbf{Optymalizacja systemu} -- znalezienie optymalnej konfiguracji 
    parametrów oraz zaprojektowanie skutecznych promptów dla różnych typów pytań.
    
    \item \textbf{Analiza wyników} -- porównanie wpływu poszczególnych czynników 
    na jakość systemu i sformułowanie wniosków dla przyszłych implementacji.
\end{enumerate}



\section{Thesis Outline}
Zarysuj strukturę swojej pracy dyplomowej. Ogólnie przedstawienie pracy. Przykładowo: ,,Praca dzieli się na $7$ rozdziałów (\dots)''. Rozdział \ref{chapter:politechnika} dotyczy (\dots). Temat został rozwinięty~w~\ref{chapter:podrozdzial}.